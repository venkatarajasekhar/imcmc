\documentclass[11pt,a4]{paper}
\usepackage{bm}
\usepackage{caption}
\usepackage{algorithm}
\usepackage{algpseudocode}
\usepackage{xcolor}
\usepackage{amsmath}
\usepackage{geometry}
\usepackage{calc}
\usepackage[colorlinks,linkcolor=blue,citecolor=blue,urlcolor=blue]{hyperref}
\usepackage{multicol}

%	set text width and heights
\setlength\textwidth{7in}
\setlength\textheight{9in}
\setlength\oddsidemargin{(\paperwidth-\textwidth)/2 - 1in}
\setlength\topmargin{(\paperheight-\textheight-\headheight-\headsep-\footskip)/2 - 1 in}


\newcommand{\hide}[1]{}


\title{How to use IMCMC}
\author{Youhua Xu\\
	yhxu@nao.cas.cn\\
	NJU $\&$ NAOC}



\begin{document}
\maketitle

\abstract{Observational cosmology is my working area.  Constraining cosmological models using
various observational datasets is a hot topic, in which MCMC simulations are often involved.  The
problems encounterred in cosmological are rather complicated, the \textit{likelihood} functions
are also complicated and live in very high dimensions, thus a robust MCMC sampler is extreamly
important to efficient problem solving, that's what IMCMC trying to be.  This document explains
the code structure of IMCMC and gives some examples on how to use it.}

\tableofcontents

\begin{multicols}{2}

\section{Basics of MCMC}

\section{Robust algrithms}
This section introduce my favorite MCMC algrithms/samplers, but currently only the
Afine-invariant ensemble sampler [\ref{sec:ensemble}] has been implemented in IMCMC.

\subsection{Afine-invariant ensemble sampler} \label{sec:ensemble}

\subsection{Nest sampler}

\subsection{Hamiltonian sampler}

\section{What should IMCMC be capable of?}

\subsection{Parsering parameters from $*.ini$ files}
Usually one will use may parameters in her/his codes, which might be model parameters, names/paths
of data and even precision controlling parameters, thus a well-designed parser is very helpful.

\subsection{Recording informations}

\section{Structure of IMCMC}


\end{multicols}
\end{document}
